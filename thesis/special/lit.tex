%%%%%%%%%%%%%%%%%%%%%%%%%%%%%%%%%%%%%%%%%%%%%%%%%%%%%%%%%%%%%%%%%%%%%%%%%%%%%
\begin{thebibliography}{}                               
	 \label{literatura}
%\bibitem{urlZdroj}\url{https://github.com/chovancova/diploma-thesis-fuzzification} 

%\bibitem{Priezvisko} Priezvisko M., {\it Názov knihy, clanku} , 
%Vydavatelstvo, 2013, ISBN 978-1-449-36536-3

\bibitem{levashenkoProj} LEVASHENKO V. - ZAITSEVA E. - KOVALÍK Š. {\it Projektovanie systémov pre podporu rozhodovania na základe neurčitých dát}. Žilinská univerzita v Žiline/EDIS, 2013. ISBN 978-80-554-0680-0.

%todo zjednotit bodky a ciarky v referenciach

\bibitem{Zadeh1965} Zadeh L., {\it Fuzzy sets}. Information and Control, vol.8, 1965, pp. 338-353. 

\bibitem{Kaufman1985} Kaufmann A., Gupta M., {\it Induction to fuzzy arithmetic: theory and applications}. New York : Van Nostrand Reinold Co., 1985, 361 p. 
\bibitem{Klir1995}
Klir G., Yuan B., {\it Fuzzy Sets and Fuzzy Logic. Theory and Applications}. Prentice Hall, 1995, 591 p. 
\bibitem{Navara2011}
Navara M., {\it Computation with fuzzy quantities}. Proc. of the 7th Conf. of the European Society for Fuzzy Logic and Technology (EUSFLAT), Aix-les-Bains, France, 2011, pp. 209-214.






\bibitem{gregorUI} GREGOR M., {\it Umelá inteligencia 1} , CEIT, 2014, ISBN 978-80-971684-1-4.
%kniha o umelej inteligencii = jeho prva referencia
\bibitem{gregorref1} SPALEK.J - JANOTA. A - BLAŽOVIČOVÁ, M. - PŘIBYL, P. {\it Rozhodovanie a riadenie s podporou umelej inteligencie}. Žilinská univerzita v Žiline/EDIS, 2005. ISBN 80-8070-354-X. 

\bibitem{gregorRef13} NICKLES. M - SOTTARA, D. {\it Approaches to Uncertain or Imprecise Rules - A Survey}. In Rule Interchange and Applications, vol. 5858 of {\it Lecture Notes in Computer Science}, pp. 323-336. Springer, 2009. ISBN 9783642049842. 

%pozri a najdi tento zdroj - Ross, T.J. Fuzzy Logic with Engineering Applications. .. alebo pozri zdroje v knihe 14
\bibitem{gregorRef14} ROSS, T.J. {\it  Fuzzy Logic with Engineering Applications.}. John Wiley \& Sons, 2004, second edition ed. ISBN 0-470-86075-8. 



\bibitem{gregorRef19}PASISNO, K. M. - YURKOVICH, S. {\it Fuzzy control}, vol.42. Addison Wesley Longman, 1998. ISBN 0-201-18074-X.  


\bibitem{Garcia2013} García S., Luengo J. Sáes J.A., López V., Herrera F., \textit{ A Survey of Discretization Techniques: 	Taxonomy and Empirical Analysis	in Supervised Learning }
IEEE TRANSACTIONS ON KNOWLEDGE AND DATA ENGINEERING, VOL. 25, NO. 4, 2013, pp. 734-750.




%Wong1987
\bibitem{Wong1987} 
Wong A.K., Chiu D.K., {\it Synthesizing Statical Knowledge from Incomplete Mixed-Mode Data}. IEE Trans. on Pattern Analysiss and Machine Intelligence, vol.9, no.6, 1987, pp. 796-805. 

%Keber1992
\bibitem{Keber1992} 
Kerber R., {\it ChiMerge: Discretitation of numeric attributes}. Proc. of the 9th National Conf. on Artifical Intelligence American Association for Artifical Intelligence, 1992, pp.123-128. 

%Quinlan1993
\bibitem{Quinlan1993} 
Quinlan J.R, {\it C4.5: Programs for Machine Learning}. Morgan Kaufmann Publ. Inc, San Manteo, California, 1993. 


%Fayyad1993
\bibitem{Fayyad1993} 
Fayyad U.M., Irani K.B., {\it Multi-Interval Discretitation of Continuous-Valued Attributes for Classification Learning}, 
Proc. of the 13th Int. Joint Conf. on Artifical Intelligence, 1993, pp. 1022-1027. 


%Ventura1994
\bibitem{Ventura1994} 
Ventura D., Martinez T. R., {\it BRACE: A paradigm for the discretization of Continuously valued data. } 
Proc. of the 7th Annual Florida AI Research Symposium (FLAIRS), 1994, pp. 117-121. 


%Pazzani1995
\bibitem{Pazzani1995} 
Pazzani M.J., {\it An iterative improvement approach for the discretization of numeric attributes in bayesian classifiers}.
Proc. of the Int. Conf. on Knowledge Discovery and Data Mining (KDD), 1995, pp. 228-233. 


%Ching1995
\bibitem{Ching1995} 
Ching J.Y., Wong A.K.C., Chan, K.C.C., {\it Class-Dependent Diskretization for Inductive Learning from Continuous and Mixed-Mode Data}.
IEEE Trans. on Pattern Analysis and Machine Intelligence, vol.17, no.7, 1995, pp. 641-651.


%Liu1997
\bibitem{Liu1997} 
, {\it  }


%Ho1997
\bibitem{Ho1997} 
, {\it  }


%Monti1998
\bibitem{Monti1998} 
, {\it  }


%Grzymala2001
\bibitem{Grzymala2001} 
, {\it  }


%Bay2001
\bibitem{Bay2001} 
, {\it  }


%Giraldez2002
\bibitem{Giraldez2002} 
, {\it  }


%Kurgan2004
\bibitem{Kurgan2004} 
, {\it  }


%Chao2005
\bibitem{Chao2005} 
, {\it  }


%Mehta2005
\bibitem{Mehta2005} 
, {\it  }


%Ferrandiz2005
\bibitem{Ferrandiz2005} 
, {\it  }


%Boulle2006
\bibitem{Boulle2006} 
, {\it  }


%Au2006
\bibitem{Au2006} 
, {\it  }


%Lee2007
\bibitem{Lee2007} 
, {\it  }


%Flores2007
\bibitem{Flores2007} 
, {\it  }


%Ruiz2008
\bibitem{Ruiz2008} 
, {\it  }


%Hu2008
\bibitem{Hu2008} 
, {\it  }


%Berrado2009
\bibitem{Berrado2009} 
, {\it  }


%Jiang2009
\bibitem{Jiang2009} 
, {\it  }


%Jin2009 - založený na pojmoch teórie informácie. 
\bibitem{Jin2009} 
, {\it  }


%Yang2009a
\bibitem{Yang2009a} 
, {\it  }


%Jiang2009
\bibitem{Jiang2009} 
, {\it  }


%Nemmiche2010
\bibitem{Nemmiche2010} 
, {\it  }


%Zhu2010
\bibitem{Zhu2010} 
, {\it  }


%Li2010a
\bibitem{Li2010a} 
, {\it  }


%Gupta2010
\bibitem{Gupta2010} 
, {\it  }


%Sang2010
\bibitem{Sang2010} 
, {\it  }


%Garcia2010
\bibitem{Garcia2010} 
, {\it  }



        
\end{thebibliography}

