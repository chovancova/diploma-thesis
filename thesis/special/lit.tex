%%%%%%%%%%%%%%%%%%%%%%%%%%%%%%%%%%%%%%%%%%%%%%%%%%%%%%%%%%%%%%%%%%%%%%%%%%%%%
\begin{thebibliography}{}                               
	 \label{literatura}

%\bibitem{Priezvisko} Priezvisko M., {\it Názov knihy, clanku} , 
%Vydavatelstvo, 2013, ISBN 978-1-449-36536-3

\bibitem{levashenkoProj} LEVASHENKO V. - ZAITSEVA E. - KOVALÍK Š., {\it Projektovanie systémov pre podporu rozhodovania na základe neurčitých dát}. Žilinská univerzita v Žiline/EDIS, 2013. ISBN 978-80-554-0680-0.

%todo zjednotit bodky a ciarky v referenciach

\bibitem{Zadeh1965} Zadeh L., {\it Fuzzy sets}. Information and Control, vol.8, 1965, pp. 338-353. 

\bibitem{Kaufman1985} Kaufmann A., Gupta M., {\it Induction to fuzzy arithmetic: theory and applications}. New York : Van Nostrand Reinold Co., 1985, 361 p. 

\bibitem{Klir1995} Klir G., Yuan B., {\it Fuzzy Sets and Fuzzy Logic. Theory and Applications}. Prentice Hall, 1995, 591 p. 

\bibitem{Navara2011} Navara M., {\it Computation with fuzzy quantities}. Proc. of the 7th Conf. of the European Society for Fuzzy Logic and Technology (EUSFLAT), Aix-les-Bains, France, 2011, pp. 209-214.

\bibitem{gregorUI} GREGOR M., {\it Umelá inteligencia 1} , CEIT, 2014, ISBN 978-80-971684-1-4.
%kniha o umelej inteligencii = jeho prva referencia
\bibitem{gregorref1} SPALEK.J - JANOTA. A - BLAŽOVIČOVÁ, M. - PŘIBYL, P. {\it Rozhodovanie a riadenie s podporou umelej inteligencie}. Žilinská univerzita v Žiline/EDIS, 2005. ISBN 80-8070-354-X. 

\bibitem{gregorRef13} NICKLES. M - SOTTARA, D. {\it Approaches to Uncertain or Imprecise Rules - A Survey}. In Rule Interchange and Applications, vol. 5858 of {\it Lecture Notes in Computer Science}, pp. 323-336. Springer, 2009. ISBN 9783642049842. 

%pozri a najdi tento zdroj - Ross, T.J. Fuzzy Logic with Engineering Applications. .. alebo pozri zdroje v knihe 14
\bibitem{gregorRef14} ROSS, T.J. {\it  Fuzzy Logic with Engineering Applications.}. John Wiley \& Sons, 2004, second edition ed. ISBN 0-470-86075-8. 


\bibitem{gregorRef19}PASISNO, K. M. - YURKOVICH, S. {\it Fuzzy control}, vol.42. Addison Wesley Longman, 1998. ISBN 0-201-18074-X.  

\bibitem{Catlett1991} Catlett J., \textit{On Changing Continuous Attributes into Ordered Discrete Attributes.} Lecture Notes on Computer Science, Berlin: Springer-Verlag, vol. 482, 1991, pp. 164-177.  

\bibitem{Lui2002} Liu H., Hussain F., Lim Tan C., Dash M., \textit{Discretization: An Enabling Technique}. Data Mining and Knowledge Discovery, Kluwer Academic Publishers, vol.6, 2002, pp. 293-423. 



\bibitem{kosko25} B. Kosko, \textit{Fuzzy entropy and conditioning}, Inform. Sci., vol. 40, pp. 165-174, Dec. 1986. 
\bibitem{renyi30} A. Renyi, \textit{On the measure of entropy and information}, in Proc. Fourth Berkeley Symp. Math. Statistics Probabilit, vol. 1, Berkeley, CA, 1961, pp. 541-561. 

\bibitem{belahut31} R. E. Belahut, \textit{Principles and Practice of Information Theory}, Reading, MA: Addison-Wesley, 1987. 
\bibitem{cover33} J.Y. Ching et al., \textit{Class-dependent discretization for inductive learning form continuous and mixed-mode data}, IEEE Trans. Pattern Anal. Machine Intell., vol. 17, pp. 641-651, July 1995.  

%http://opac.crzp.sk/openURL?stype=0&sid=A61021784F381E68CE068F90830C
%Fuzzy riadenie synchrónneho motora
%Autor - Sedláková, Viera
%Školiteľ - Girovský, Peter
%Oponent - Timko, Jaroslav
%Škola - Technická univerzita v Košiciach 1040 104006
%Rok odovzdania 2013 Počet strán113s.

\bibitem{sedlakova} Autor: Sedláková, V., Školiteľ: Girovský P., Oponent: Timko J. \textit{Fuzzy riadenie synchrónneho motora}, Technická univerzita v Košiciach, Rok odovzdania 2013, Počet strán 113s. 



\bibitem{Garcia2013} García S., Luengo J. Sáes J.A., López V., Herrera F., \textit{ A Survey of Discretization Techniques: 	Taxonomy and Empirical Analysis	in Supervised Learning }
IEEE TRANSACTIONS ON KNOWLEDGE AND DATA ENGINEERING, VOL. 25, NO. 4, 2013, pp. 734-750.

\bibitem{Liu2004} Liu L., Wong A., Wang Y., \textit{A global optimal algorithm for class-dependent discretization of continuous data}. Intelligent Data Analysis, vol. 8, 2004, pp. 151-170. 

\bibitem{Singh2007} Singh G.P., Singh B., \textit{Simulink Library Development and Implementation for VLSI Testing in Matlab}, Communications in Computer and Information Science, vol.169 CCIS, 2011, pp. 233-240. 

\bibitem{Bakar2009} Bakar A., Othman Z., Shuib N., \textit{Building a New Taxonomy for Data Discretization Techniques}, Proc. on Int. Conf. on Data Mining and Optimization (DMO), 2009, pp. 132-140. 

\bibitem{Yang2010} Yang Y., Webb G. I., Wu X., \textit{Discretization methods}, Data Mining and Knowledge Discovery Handbook, 2010, pp. 101-116. 

%%Garcia2010
\bibitem{Garcia2010} Garcia M., Lucas J. P., Batista V. F. L., \textit{Multivariate discretization for associative classification in a sparse data application domain}, Proc. of the $5^th$ Int. Conf. on Hybrid Artificial Intelligent Systems (HAIS), 2010, pp. 104-111. 

\bibitem{Chelbus1998} Chelbus B., Nguyen S.H., \textit{On Finding Optimal Discretizations for Two Attributes}, Lecture Notes in Artificial Intelligence, vol.1424, 1998, pp. 537-544. 


%Wong1987
\bibitem{Wong1987} 
Wong A.K., Chiu D.K., {\it Synthesizing Statical Knowledge from Incomplete Mixed-Mode Data}. IEE Trans. on Pattern Analysiss and Machine Intelligence, vol.9, no.6, 1987, pp. 796-805. 

%Keber1992
\bibitem{Keber1992} 
Kerber R., {\it ChiMerge: Discretitation of numeric attributes}. Proc. of the 9th National Conf. on Artifical Intelligence American Association for Artifical Intelligence, 1992, pp.123-128. 

%Quinlan1993
\bibitem{Quinlan1993} 
Quinlan J.R, {\it C4.5: Programs for Machine Learning}. Morgan Kaufmann Publ. Inc, San Manteo, California, 1993. 


%Fayyad1993
\bibitem{Fayyad1993} 
Fayyad U.M., Irani K.B., {\it Multi-Interval Discretitation of Continuous-Valued Attributes for Classification Learning}, 
Proc. of the 13th Int. Joint Conf. on Artifical Intelligence, 1993, pp. 1022-1027. 


%Ventura1994
\bibitem{Ventura1994} 
Ventura D., Martinez T. R., {\it BRACE: A paradigm for the discretization of Continuously valued data. } 
Proc. of the 7th Annual Florida AI Research Symposium (FLAIRS), 1994, pp. 117-121. 


%Pazzani1995
\bibitem{Pazzani1995} 
Pazzani M.J., {\it An iterative improvement approach for the discretization of numeric attributes in bayesian classifiers}.
Proc. of the Int. Conf. on Knowledge Discovery and Data Mining (KDD), 1995, pp. 228-233. 


%Ching1995
\bibitem{Ching1995} 
Ching J.Y., Wong A.K.C., Chan, K.C.C., {\it Class-Dependent Diskretization for Inductive Learning from Continuous and Mixed-Mode Data}.
IEEE Trans. on Pattern Analysis and Machine Intelligence, vol.17, no.7, 1995, pp. 641-651.

\bibitem {lee2001} Lee H-M., Chen C.-M., Chen J.-M., Jou Y.-L., \textit{An Efficient Fuzzy Classifier with Feature Selection Based on Fuzzy Entropy.} IEEE Trans. on Systems, Man and Cybernetics, Part B: Cybernetics, vol. 31, no.3, 2001, pp. 426-432. Dostupné online: \url{http://dx.doi.org/10.1109/3477.931536}

\bibitem{chi36} Z. Chi and H. Yan, \textit{Feature evaluation and selection based on an entropy measure with data clustering}, Opt. Eng., vol. 34, pp. 3514-3519, Dec. 1995. 
\bibitem{duda37} R. O. Duda and P. E. Hart, \textit{Pattern Classification and Scene Analysis}, New York: Wiley, 1973. 

\bibitem{cheng2006} Lin C.J., Lee C.Y., Hong S.J., \textit{An Efficient Fuzzy Classifier Based on Hierarchical Fuzzy Entropy}, International Journal of Information Technology, Vol. 12, No.6, 2006.  

\bibitem{Bezdec1981} Bezdec, J.C., {\it Pattern Recognition with Fuzzy Objective Function Algorithms}, Plenum
Press, New York, 1981.

\bibitem{fuzzyLogicToolbox} Fuzzy Logic Toolbox User's Guide. Dostupné online: \url{https://www.mathworks.com/help/pdf_doc/fuzzy/fuzzy.pdf}

\bibitem{bache2013} Bache K., Lichman M., \textit{UCI Machine Learning Repository}, Irvine, CA: University of California, School of Information and Computer Science, 2013. Dostupné online: \url{http://archive.ics.uci.edu/ml}

        
\end{thebibliography}

