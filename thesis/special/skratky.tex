\chapter*{Zoznam skratiek}
\begin{acronym}
\acro{Bayesian }{Bayesian Discretizer } 
\acro{BRDisc }{Boolean Reasoning Discretizer } 
\acro{CADD}{Class-Attribute Dependent Discretizer } 
\acro{CAIM }{Class-Attribute Interdependence Maximization } 
\acro{EBDA }{Effective Botton-up Discretizer } 
\acro{EqualFrequency}{Equal Frequency Discretizer } 
\acro{EqualWidth}{Equal Width Discretizer}
\acro{FCM }{Fuzzy c-means } 
\acro{FEBFC}{Fuzzy entropy-based fuzzy classifier}
\acro{ID3}{Iterative Dichotomizer 3 Discretizer } 
\acro{MDL-Disc }{Minimum Description Length Discretizer } 
\acro{MDLP}{Minimum Description Length Principe} 
\acro{StatDisc}{Class-driven Statistical Discretizer} 
\acro{WEDA }{Wrapper Estimation of Distribution Algorithm } 

\end{acronym}

%http://opac.crzp.sk/?fn=docviewChild5
%Fuzzy riadenie synchrónneho motora [Diplomová] = Fuzzy control of synchronous motor / Bc.Viera Sedláková. - Košice, 2012. - 113 s.19. strana
\chapter*{Zoznam termínov}
\begin{acronym}
%\acro{Báza pravidiel }{obsahuje všetky fuzzy pravidlá, ktoré popisujú správanie fuzzy regulátora. } 
\acro{Defuzzifikácia }{je prevod fuzzy množín na ostrú hodnotu.} 
\acro{Funkcia príslušnosti}{charakteristická funkcia fuzzy množín, ktorá charakterizuje stupeň, s akým daný prvok patrí do príslušnej množiny a to v rozsahu od 0 do 1. }
\acro{Fuzzifikácia }{ je prevod ostrých vstupných hodnôt do fuzzy množín pomocou funkcií príslušnosti. } 
\acro{Fuzzy logika}{ vychádza z teórie fuzzy množín a zameriava sa na vágnosť, ktorou sa snaží matematicky zachytiť. } 
\acro{Fuzzy množina}{je množina, ktorej prvok patrí do množiny s istou pravdepodobnosťou a tou je stupeň príslušnosti. } 
\acro{Fuzzy}{znamená neostrý, matný, mlhavý, neurčitý, vágny.} 
\acro{Lingvistická premenná }{je premenná, ktorej hodnoty sú výrazy nejakého jazyka. }
\acro{Termy }{je množina lingvistických hodnôt. } 
\acro{Univerzum }{je univerzálna množina, na ktorej sú definované termy.}
\acro{Vágnosť }{súvisí s naším vnímaním okolitého sveta spôsobom, ako vidíme jednotlivé objekty a ich vlastnosti,na základe ktorých vytvárame pojmy.   } 
\end{acronym}
