% !TeX spellcheck = en_US
\chapter{Draft - todo - spracovanie jednotlivých kapitol} 

\section*{FEBFC algoritmus}
FEBFC algoritmus pozostáva z týchto krokov: 
\begin{description}
	\item[Krok 1.] Určenie počtu intervalov. 	
	\item[Krok 2.] Určenie polohy intervalov. 
	\item[Krok 3.] Priradenie funkcie príslušnosti pre každý interval.   
	\item[Krok 4.] Vypočítanie fuzzy entropie pre každú položku cez sumarizáciu fuzzy entropie pre všetky intervaly pre dané dimenzie položky.   
\end{description}


\subsection*{Krok 1. Určenie počtu intervalov }
Počet intervalov pre každú dimenziu má účinok na učiacu efektívnosť a klasifikačnú presnosť. Ak je počet intervalov príliš veľký, tak to zaberie veľa času na dokončenie trénovania a klasifikačného procesu a môže vzniknúť preučenie. Na druhú stranu, ak je počet intervalov príliš malý, veľkosť pre každú rozhodovaciu oblasť môže byť príliš veľká pre danú distribúciu vstupných vzorov, a klasifikačný výkon môže byť pomalší. 
 


Kroky na určenie počtu intervalov pre každú dimenziu sú nasledovné: 
\begin{description}
	\item[Krok 1.]  	
	\item[Krok 2.] 
	\item[Krok 3.] 
	\item[Krok 4.]   
\end{description}