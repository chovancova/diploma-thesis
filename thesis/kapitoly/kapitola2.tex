\chapter{Analýza existujúcich algoritmov pre fuzzifikáciu numerických hodnôt} 

\section{Fuzzy klasifikátor s možnosťou výberu na základe fuzzy entropie }

Táto kapitola popisuje efektívny fuzzy klasifikátor s možnosťou výberu založenom na meraní fuzzy entropie (FEBFC).
Fuzzy entropia je použitá na vyhodnotenie informácie o distribúcii vzorov v priestore vzorov. S touto informáciou vedia rozdeliť priestor vzorov na disjunktné rozhodovacie regióny pre rozoznávanie vzorov. Vďaka tomu, že rozhodovacie regióny sú disjunktné, aj komplexnosť, aj výpočtová náročnosť je zredukovaná, a tým pádom aj čas trénovania a klasifikácie je extrémne krátka. Hoci rozhodovacie regióny sú rozdelené do disjunktných pod priestorov, môžu dosiahnuť kvalitnú klasifikáciu vďaka tomu, že pod priestory boli správne stanovené navrhovaným meraním fuzzy entropie. Okrem toho môžem skúmať ďalšie využitie fuzzy entropie na vybraté prvky. Procedúra výberu prvkov nielenže znižuje dimenziu problému, ale aj redukuje šum, zbytočné a nedôležité prvky. 
\subsection{Opis upravenej fuzzy entropie}

todo - word
\subsection{Algoritmus FEBFC}
Algorimus FEBFC sa skladá z nasledujúcich krokoch: 
\begin{description}
	\item[Krok 1.] Zistenie počtu intervalov pre každú dimenziu. 
	\item[Krok 2.] Zistenie centra a šírku pre každý interval.
	\item[Krok 3.] Priradenie funkcie príslúšnosti pre každý interval. 
	\item[Krok 4.] Označenie tried pre každý rozhodovací región.   

\end{description}

\section{Minimum description length partition}	

1. MDLP method developed in the
Fayyad, U. M. and Irani, K. B. (1993). Multi-interval discretization of continuous-valued attributes for classification learning, Artificial intelligence, 13, 1022-1027.
I had an interactive version of the program, which lets you choose from several stopping criteria:
1) using the criteria proposed in the original work;
2) criteria for the number of partitions intevalov;
3) criteria for the threshold Gini index (which assesses the effectiveness of the partition at the point in terms of the decrease in entropy).

\subsection{Algoritmus MDLP}

\section{Záver}