\chapter{Teoretické východiská práce} 

\section{Získavanie znalostí z databáz }

\section{Fuzzy logika }

\section{Meranie Entropie}
Entropia je meraná množstvom neistoty výsledku náhodného experimentu, alebo equivalene, meraním informácií keď sa pozoruje výsledok. Tento koncept bol zadefinovaný rôznymi spôsobmi [25]–[30] a zovšeobecnený v rozličných aplikovaných oblastiach, ako napríklad teória komunikácie, matematiky, štatistickej termodynamike a ekonómii [31]–[33]. Z pomedzi týchto rozličných definícií, Shannon prispel k najširšej a najfundamentálnejšej definícii entropie v informačnej teórii. V nasledujúcom texte najprv uvedieme Shannonovu entropiu a potom popíšeme štyry Luca-Termini axiómy [25], ktoré dobre-definovaná entropia musí spĺňať. Nakoniec navrhneme meranie fuzzy entropie, ktoré je rozšírenie Shannonovej definície.

\subsection{Shannonova Entropia}
Za entropiu možno považovať meranie neistoty náhodnej premennej X. Nech X je náhodná spočítateľná premenná s konečnou N-znakovou abecedou danou   .
Ak výsledok $x_j$ sa vyskytuje s pravdepodobnosťou p($x_j)$, tak potom množstvo informácie spojené so známim výskytom výstupu $x_j$ je definované ako:

1. TODO
To znamená, že pre diskrétne zdroje, informácie získané výberom symbolu sú bitové. V priemere, symbol  bude vybratý -krát z celkového počtu N výberov, takže priemerné množstvo informácie získanej z nzdrojových výsledkou je:

2. TODO 

Podelením (2.) číslom n získame priemerné množstvo informácie na symbol výstubu zdroja. To je známe ako priemerná informácia, neistota, alebo entropia definovaná nasledovne.
Definícia 1:  Entropia H(X) náhodnej diskrétnej premennej X je definovaná ako 

3. TODO
Alebo
4.	TODO
Kde
Všimnite si, že entropia je funkcia distribúcie X. Nezáleží na skutočných hodnotách náhodnej premennej X, ale iba na pravdepodobnostiach. Preto entropiu možno zasísať ako H(p).


\section{TODO INE }

\section{Záver}