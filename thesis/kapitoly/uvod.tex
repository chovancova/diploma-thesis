\chapter*{Úvod}
\addcontentsline{toc}{chapter}{Úvod}
V súčasných systémoch pre podporu rozhodovania sú využívané rôzne metódy a algoritmy. Tie by mali brať do úvahy nestochastickú neurčitosť vstupných dát. Toto je spôsobené nedostatočnou presnosťou merania dát.  Neurčité fuzzy dáta je vyjadrenie vstupných dát, ktoré berie do úvahy práve neurčitosť dát. Spracovanie neurčitých a viachodnotových dát je vyjadrené pomocou matematicky cez fuzzy a viac-hodnotovú logiku.  

Lingvistické premenné sú často používaným spôsobom na vyjadrenie neurčitých dát v konečnej množine. Výsledok formalizácie expertných odhadov sú kvalitatívne veličiny. Každý objekt, proces je opísaný skupinou vlastností, ukazovateľov. Reálne čísla sú často používané v modeloch pre podporu rozhodovania. Použitie reálnych čísiel je zložité na presné meranie hodnôt ukazovateľa. Ďalší problém je v dostupnosti nameraných dát. Meranie presných hodnôt ukazovateľov je spojené s relatívne vysokými nákladmi. V súčastnosti nie sú algoritmy a metódy, ktoré vypočítajú presné hodnoty ukazovateľov. Niekedy reálne hodnoty obsahujú zbytočne podrobnú informáciu, ktorá nemôže byť použitá ako základ na výber rozhodnutia. 

Lingvistický prístup je dobrý v tom, že umožňuje formalizovať neurčité fuzzy pojmy pomocou fuzzy množín a premenných a následne ich spracovávať cez teóriu fuzzy logiky. 

Lingvistická premenná sa od číselnej premennej líši tým, že jej hodnotami nie sú čísla, ale slová alebo výroky prirodzeného alebo formálneho jazyka. Je zrejmé, že takýto kvalitatívny popis s využitím slov je menej presný ako pomocou čísel. Napriek tomu použitie lingvistickej premennej umožňuje približne opísať zložité javy, ktoré nie je možné opísať pomocou obvyklých kvalitatívnych termínov. \cite{levashenkoProj}.


\section*{Cieľ práce}
Cieľom diplomovej práce je experimentálne porovnať algoritmy, ktoré slúžia pre fuzzifikáciu numerických hodnôt a prípadne ich modifikovať.

\subsection*{Postup práce }
\begin{enumerate}
\item Oboznámenie sa s problematikou fuzzifikácie (transformácie numerických hodnôt na lingvistické).
\item Rozbor existujúcich algoritmov pre fuzzifikáciu numerických hodnôt.
\item  Implementácia vybraných algoritmov fuzzifikácie v jazyku C++.
\item  Experimentálne porovnanie implementovaných algoritmov na rôznych výstupných dátach.
\end{enumerate}

















